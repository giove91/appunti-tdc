\chapter{Definizioni}


\section{Macchine di Turing deterministiche}
\label{sec:det-TM}
Una macchina di Turing deterministica, detta anche semplicemente macchina di
Turing, può essere immaginata come un dispositivo libero di
muoversi su un nastro infinito sul quale in ogni istante possono essere scritti
finiti simboli appartenenti ad un alfabeto finito. La macchina di Turing è
dotata di uno \emph{stato}, e lavora nel modo seguente:
\begin{enumerate}
 \item Legge il simbolo scritto sul nastro nella posizione in cui si trova attualmente.
 \item Controlla in una tabella finita memorizzata al suo interno cosa deve fare
 nel quando legge quel simbolo nello stato in cui si trova.
 \item Basandosi su quello che ha letto nella tabella scrive un simbolo sul nastro,
 si sposta a destra o a sinistra e passa in un nuovo stato.
 \item Se lo stato in cui si trova ora è uno stato speciale, chiamato \emph{terminale},
 la macchina si ferma. Altrimenti il processo reinizia.
\end{enumerate}
All'istante zero la macchina si trova in uno stato speciale detto \emph{iniziale}
e sul nastro può essere già scritta una stringa finita detta \emph{input} della macchina.
Diamo ora la definizione formale.
\begin{definizione}
Si dice macchina di Turing una tupla
$M=\langle Q, \Gamma, b, \delta, q_0, F, F_y \rangle$ dove:
\begin{enumerate}
 \item $Q$ è un insieme finito non vuoto di \emph{stati};
 \item $\Gamma$ è un insieme finito di simboli, detto \emph{alfabeto} della macchina;
 \item $b \in \Gamma$ è un simbolo speciale, detto \emph{blank}, che rappresenta
 una cella vuota;
 \item $q_0 \in Q$ è lo \emph{stato iniziale};
 \item $g: Q \to \{\exists, \texttt{accept}, \texttt{reject}\}$
  indica per ogni stato se questo è \emph{non terminale}, \emph{terminale accettante} o
  \emph{terminale rifiutante} (l'uso del simbolo ``$\exists$'' è per compatibilità
  con la Definizione \ref{def:alt-TM}).
 %\item $F \subseteq Q$ è un insieme di \emph{stati terminali};
 %\item $F_y \subseteq F$ è un insieme di stati terminali \emph{accettanti};
 \item $\delta: Q \times \Gamma \to Q \times (\Gamma \setminus \{b\}) \times \{\texttt{L},\texttt{R},\texttt{S}\}$
 è una funzione parziale detta \emph{funzione di transizione}.
\end{enumerate}
\end{definizione}

\begin{osservazione}
 Con questa definizione abbiamo scelto di non permettere alla macchina di Turing
 di cancellare una cella (ovvero di scrivere il simbolo blank).
 Questo non modifica la potenza della macchina e permette di definire più facilmente
 lo spazio usato.
\end{osservazione}


\begin{definizione}
 Data una macchina di Turing $M=\langle Q, \Gamma, b, \delta, q_0, g \rangle$,
 chiamiamo configurazione della macchina una tripla
 $\langle q, S, n \rangle$ dove:
 \begin{enumerate}
  \item $q \in Q$ rappresenta lo stato in cui si trova attualmente la macchina;
  \item $S \in \Gamma ^ \Z$, con $S(k)=b$ per tutti tranne al più finiti $k \in \Z$,
  rappresenta lo stato attuale del nastro;
  \item $n \in \Z$ rappresenta la posizione attuale della macchina di Turing
  sul nastro.
 \end{enumerate}
\end{definizione}
Nel seguito useremo la seguente notazione più compatta per descrivere la
configurazione di una macchina di Turing:
\begin{definizione}
\label{def:conf-TM}
 Data una macchina di Turing $M=\langle Q, \Gamma, b, \delta, q_0, g \rangle$,
 chiamiamo configurazione della macchina una funzione
 $S: \Z \to (\Gamma \sqcup \Gamma \times Q)$  tale che $S(k)=b$ per
 tutti tranne al più finiti $k \in \Z$ ed esiste unico $n \in \Z$ con
 $S(n) = (\gamma, q) \in \Gamma \times Q$.
 $S$ rappresenta lo il contenuto attuale del nastro, $n$ la posizione della
 macchina di Turing e $q$ il suo stato nella configurazione $S$.
\end{definizione}

\begin{definizione}
 Sia data una macchina di Turing $M=\langle Q, \Gamma, b, \delta, q_0, g \rangle$
 e due sue configurazioni $S$ e $S'$.
 Siano $n \in \Z$, $\gamma \in \Gamma$, $q \in Q$ gli unici tali che
 $S(n) = (\gamma, q)$ e supponiamo $\delta(\gamma, q) = (q', \gamma', \texttt{R})$.
 Diciamo che
 $M$ può fare la transizione da $S$ a $S'$, e lo indichiamo con $S \vdash_M S'$,
 se 
 \begin{enumerate}
  \item $S'(n)=\gamma'$;
  \item $S'(n+1)=(\sigma, q')$ per un
 qualche $\sigma \in \Gamma$;
  \item $S(k) = S'(k)$ per ogni $k \in \Z$ diverso da
 $n$ e $n+1$.
 \end{enumerate}
 La definizione si adatta facilmente ai casi in cui
 $\delta(\gamma, q) = (q', \gamma', \texttt{L}/\texttt{S})$.
\end{definizione}

\begin{osservazione}
 Per ogni configurazione $S$ della macchina di Turing $M$, esiste al più una
 $S'$ tale che $S \vdash_M S'$, da cui l'aggettivo \emph{deterministica}.
\end{osservazione}

% \begin{definizione}
%  Data una macchina di Turing $M$ e due sue configurazioni $S$ e $S'$ diciamo
%  che $S \vdash_M^{0} S'$ se $S = S'$ e diciamo che
%  $S \vdash_M^{n+1} S'$ se esiste una configurazione $S_0$ tale che
%  $S \vdash_M S_0$ e $S_0 \vdash_M^{n} S'$. Diciamo poi che $S \vdash_M^{\ast} S'$
%  se esiste $n \in \N$ tale che $S \vdash_M^{n} S'$.
% \end{definizione}

\begin{definizione}
 Una configurazione si dice accettante se lo stato corrispondente è accettante,
 oppure se può fare una transizione in una configurazione accettante.
\end{definizione}

\begin{definizione}
 Una configurazione $S$ di una macchina di Turing
 $M=\langle Q, \Gamma, b, \delta, q_0, g \rangle$ si dice \emph{iniziale} se
 $S(0) = (\gamma, q_0)$. In tal caso, il contenuto del nastro
 si dice \emph{input} della macchina di Turing e questo determina totalmente
 la configurazione iniziale.
%  Una configurazione $S$ si dice \emph{terminale} se lo stato della macchina
%  di Turing nella configurazione $S$ è terminale e si dice accettante se lo stato
%  della configurazione è accettante.
\end{definizione}

\begin{definizione}
\label{def:accept-input}
 Sia $S_w$ la configurazione iniziale con input $w$ in $\Gamma^\ast$. Diciamo che
 una macchina di Turing accetta l'input $w$, e lo indichiamo con $M(w)\downarrow_y$
 se la configurazione $S_w$ è accettante.
%  Diciamo che
%  $M$ accetta l'input $w$, e lo indichiamo con $M(w)\downarrow_y$ se
%  esiste $S'$ terminale accettante tale che $S \vdash_M^\ast S'$.
\end{definizione}

\begin{definizione}
\label{def:problem}
 Dato un alfabeto $\Gamma$, chiamiamo \emph{problema decisionale} un insieme di stringhe finite
 $P \subseteq \Gamma^\ast$. Diciamo che il problema $P$ è calcolabile dalla macchina
 di Turing $M$ se, per ogni $w$ in $\Gamma^\ast$, $M$ accetta $w$ se e solo se $w$
 è in $P$. Diciamo che il problema $P$ è calcolabile se è calcolabile da una
 macchina di Turing.
\end{definizione}

Esistono varianti della definizione di macchina di Turing che, pur non cambiano
l'insieme dei problemi calcolabili, sono più semplici da adoperare in alcuni
casi. Una variante utile sono le
\emph{macchine di Turing a $k$ nastri}. Queste funzionano esattamente come le
macchine di Turing standard ma ad ogni passaggio leggono e scrivono contemporaneamente su
tutti e $k$ i nastri. La funzione di transizione sarà dunque del tipo
$\delta:  Q \times \Gamma^k \to Q \times ((\Gamma \setminus \{b\}) \times \{\texttt{L},\texttt{R},\texttt{S}\})^k$.
Una macchina di Turing a $k$ nastri può chiaramente simulare una macchina di Turing
a un solo nastro. Vale anche il viceversa:

\begin{fatto}
Una macchina di Turing a un solo nastro può simulare una macchina di Turing
a $k$ nastri.
In particolare tutti i problemi calcolabili da una macchina
di Turing a k nastri sono calcolabili anche da una macchina di Turing ad un nastro.
\end{fatto}

Il modello di macchina di Turing che useremo più spesso è quello di una macchina di Turing
con un nastro di input e un nastro di lavoro. Questa segue le stesse regole di
una macchina di Turing a due nastri, fatta eccezione del fatto che non può scrivere
sul nastro di input, ma solo leggere, e che in una sua configurazione iniziale
il nastro di lavoro è sempre vuoto. Se una macchina di Turing di questo tipo
termina con input $w$, definiamo lo spazio di lavoro usato su input $w$
come il numero di celle del nastro di lavoro non vuote nella configurazione terminale.


\section{Macchine di Turing alternanti e non deterministiche}
Una macchina di Turing alternate può essere pensata come una normale macchina di
Turing che ad ogni transizione può ``sdoppiarsi'' e provare più strade contemporaneamente
per risolvere il problema.

\begin{definizione}
\label{def:alt-TM}
 Si dice macchina di Turing alternate una quintupla
 $M=\langle Q, \Gamma, b, \delta, q_0, g \rangle$ dove:
 \begin{itemize}
  \item $Q$ è un insieme finito non vuoto di \emph{stati};
  \item $\Gamma$ è un insieme finito di simboli, detto \emph{alfabeto} della macchina;
  \item $b \in \Gamma$ è un simbolo speciale, detto \emph{blank}, che rappresenta
  una cella vuota;
  \item $\delta: Q \times \Gamma \to \powerset(Q \times \Gamma \times \{\texttt{L},\texttt{R},\texttt{S}\})$;
  \item $q_0 \in Q$ è lo \emph{stato iniziale};
  \item $g: Q \to \{\exists, \forall, \texttt{accept}, \texttt{reject}\}$
  indica per ogni stato se questo è \emph{esistenziale}, \emph{universale},
  \emph{accettante} o \emph{rifiutante}.
 \end{itemize}
\end{definizione}

Per una macchina di Turing alternante valgono le stesse definizioni che
abbiamo dato per le macchine di Turing deterministiche, 
fatta eccezione per la definizione di transizione e di stato accettante, che
modifichiamo come segue.
\begin{definizione}
 Sia data una macchina di Turing alternante $M=\langle Q, \Gamma, b, \delta, q_0, g \rangle$
 e due sue configurazioni $S$ e $S'$.
 Siano $n \in \Z$, $\gamma \in \Gamma$, $q \in Q$ gli unici tali che
 $S(n) = (\gamma, q)$ e supponiamo $(q', \gamma', \texttt{R}) \in \delta(\gamma, q)$.
 Diciamo che  $M$ può fare la transizione da $S$ a $S'$, e lo indichiamo con
 $S \vdash_M S'$, se 
 \begin{enumerate}
  \item $S'(n)=\gamma'$;
  \item $S'(n+1)=(\sigma, q')$ per un
 qualche $\sigma \in \Gamma$;
  \item $S(k) = S'(k)$ per ogni $k \in \Z$ diverso da
 $n$ e $n+1$.
 \end{enumerate}
 La definizione si adatta facilmente ai casi in cui
 $(q', \gamma', \texttt{L}/\texttt{S}) \in \delta(\gamma, q)$.
\end{definizione}

\begin{osservazione}
 Per una generica macchina di Turing alternante non è più vero che per ogni
 configurazione $S$ esiste al più una configurazione $S'$ tale che $S \vdash_M S'$.
\end{osservazione}

\begin{definizione}
 Una configurazione di una macchina di Turing alternante si dice accettante se
 vale una delle seguenti:
 \begin{enumerate}
  \item Il suo stato è accettante;
  \item Il suo stato è esistenziale e la configurazione può fare una transizione
  in uno stato accettante;
  \item Il suo stato è universale e ogni transizione che la configurazione può fare
  finisce in uno stato accettante.
 \end{enumerate}
\end{definizione}
Possiamo infine definire cosa vuol dire che una macchina di Turing alternante
$M$ accetti un input $w$ allo stesso modo della Definizione \ref{def:accept-input}.

\begin{definizione}
 Chiamiamo \emph{macchina di Turing non deterministica} una macchina di Turing
 alternante in cui tutti gli stati sono esistenziali, accettanti o rifiutanti.
\end{definizione}

\begin{osservazione}
 Una macchina di Turing deterministica può essere vista come un caso particolare
 di una macchina di Turing non deterministica dove ogni
 stato può fare una transizione in al più un'altro stato.
\end{osservazione}

Uno strumento molto utile nello studio delle macchine di Turing alternati è
il grafo delle configurazioni $G_M$:

\begin{definizione}
 Data una macchina di  Turing alternante $M$ con un unico stato accettante,
 il grafo delle configurazioni $G_M$
 è la struttura nel linguaggio $L=\langle E, G_\exists, G_\forall, t\rangle$
 definita da:
 \begin{itemize}
  \item l'insieme $V$ dei vertici del grafo coincide con l'insieme di tutte le
  possibili configurazioni di $M$;
  \item gli archi del grafo $E$ rappresentano le possibili transizioni, ovvero
  vale $E(S,S') \iff S \vdash_M S'$;
  \item $G_\exists \subseteq V$ è un'etichetta dei nodi che indica le configurazioni
  con stato esistenziale;
  \item $G_\forall \subseteq V$ è un'etichetta dei nodi che indica le configurazioni
  con stato universale.
  \item $t \in V$ è una costante che indica l'unica configurazione avente nastro
  vuoto e stato accettante.
 \end{itemize}
\end{definizione}
Questo grafo descrive completamente tutte le possibili computazioni di $M$ con
input qualsiasi. Spesso siamo interessati a una particolare computazione
con input $w$ fissato:
\begin{definizione}
 Il grafo $G_{M,w}$
 della computazione di $M$ con input $w$ è dato dalla
 struttura $(V,E, G_\exists, G_\forall, s, t)$ ottenuta estendendo $G_M$
 con una costante $s$, che denota la configurazione iniziale con input $w$.
 Se non c'è ambiguità al posto di $G_{M,w}$ scriveremo semplicemente $G_w$.
\end{definizione}
Dato $G_w$, è interessante trovare un modo per capire se una configurazione
è accettante, in particolare se lo è la configurazione iniziale $s$, guardando
solo il grafo.
\begin{definizione}
 Dato $G_{M,w}$ il grafo di una computazione,
 definiamo $\Ealt(x,y)$ essere la
 più piccola (rispetto all'inclusione) relazione binaria su $G_{M,w}$ tale che
 \[
  \Ealt(x,y) \longleftrightarrow (x=y) \lor \exists z 
  \left[ E(x,z) \land \Ealt(z,y) \land (G_\forall(x) \rightarrow \forall z\left(E(x,z) \rightarrow \Ealt(z,y)\right)\right]
 \]
 \mytodo{Divertiti a renderla leggibile Giove!}
\end{definizione}
Che esista una minima tale relazione può essere verificato facilmente a mano, o
lo si può dedurre dal più generale Corollario \ref{cor-f-phi}.
\begin{definizione}
\label{def:Ralt}
 Data una macchina di turing $M$, $\Ralt = \{G_w | \Ealt(s,t) \}$ è l'insieme
 dei grafi che $G_w$ tali che $M$ accetta l'input $w$.
\end{definizione}


\section{Codifica e decodifica}
Sia $L=\{R^{a_1}_1, \ldots, R^{a_s}_s, c_1, \ldots, c_l\}$ un linguaggio.
Vogliamo fornire un modo standard di codificare una $L$-struttura finita $A$ come
stringa binaria, che denoteremo $\bin(A)$. Nel seguito supporremo sempre, senza
perdita di generalità, che
il dominio di una $L$-struttura finita sia un insieme del tipo
$\{1,\ldots,n\} \subseteq \N$.

\begin{definizione}
  Sia $A$ una $L$-struttura, con $|A| = n < +\infty$.
  Se $R \subseteq A^k$ è una relazione,
  definiamo $\bin(R): |A|^k \to \{0,1\}$ come:
  \[\bin(R)(a_1 + a_2 \cdot n + a_k \cdot n^{k-1}):=
    \begin{cases}
      1 \mbox{ se } R(a_1,\ldots,a_k) \\
      0 \mbox{ altrimenti}
    \end{cases}
  \]
  Se $a$ è un elemento di $A=\{1,\ldots,n\}$, definiamo $\bin(a)$ essere la
  scrittura binaria del numero $a$, con eventualmente aggiunti degli zeri a
  sinistra affinché abbia lunghezza esattamente $\lceil \log_2(n) \rceil$.
  Definiamo infine
  \[\bin(A):=\bin(R_1)\string^\ldots\string^\bin(R_s)\string^\bin(c_1)\string^\ldots\string^\bin(c_l).\]
\end{definizione}

\begin{osservazione}
\label{oss-bin}
 Detta $n$ la cardinalità di $A$, la lunghezza della codifica è
 \[ |\bin(A)|=n^{a_1} + \ldots + n^{a_s} + l \cdot \lceil \log_2(n) \rceil, \]
 e in particolare è polinomiale in $n$. Inoltre dato che la parte a destra è
 strettamente crescente in $n$, questo ci dice che, conoscendo solo il linguaggio,
 e $\bin(A)$, possiamo ricavare $n$ e più in generale possiamo ricostruire tutto $A$.
 Infatti avremo che vale $R_i^A(x_1,\ldots, x_{a_i})$ se e solo se
 $\bin(A)(n^{a_1} + \ldots + n^{a_{i-1}} + x_1 + x_2 \cdot n + \ldots + x_{a_i} \cdot n^{a_i-1}) = 1$.
\end{osservazione}
Useremo questa procedura principalmente per descrivere modelli all'interno
di macchine di Turing. Infatti, come notato nell'Osservazione \ref{oss-bin},
una macchina di Turing può ricostruire
il valore della relazione $R_i^A(x_1, \ldots, x_{a_i})$ calcolando il valore
di $n$, spostando la testina e leggendo il valore nella posizione
$n^{a_1} + \ldots + n^{a_{i-1}} + x_1 + x_2 \cdot n + \ldots + x_{a_i} \cdot n^{a_i-1}$
del blocco di memoria corrispondente alla codifica.
Nello pseudo-codice indicheremo questa procedura con
$\decode_{R_i}(\bin(A), x_1, \ldots, x_{a_i})$. La corrispondente procedura per
leggere il valore di una costante sarà indicata con $\decode_{c_i} (\bin(A))$.

\begin{definizione}
Fissato un linguaggio $L$ ci riferiremo a un insieme $\mathcal{B}$ di $L$-strutture finite
come a un \emph{problema decisionale}, intendendo con questo che consideriamo il corrispondente
problema $\bin(\mathcal{B}) = \{\bin(A) \mid A \in \mathcal{B}\}$ nel senso della Definizione \ref{def:problem}.
\end{definizione}
In particolare $\Ralt$ (cfr. Definizione \ref{def:Ralt}) è un problema, ed ha
dunque senso chiedersi se sia calcolabile e con quale complessità
(cfr. Proposizione \ref{prop:ralt-in-fo(lfp)} e seguenti).