\chapter{Caratterizzazione e proprietà di $\NP$}

\section{$\NP$ equivale ad $\ESO$}

\begin{teorema}[Fagin]
  \label{thm:np-eso}
  $\NP = \ESO$.
\end{teorema}


TODO


\section{$\ESOmon \neq \coESOmon$}

Sebbene non si sappia se $\NP\neq\coNP$, si può dimostrare un risultato simile: $\ESOmon \neq \coESOmon$.
\mytodo{Definire $\ESOmon$.}
Per fare questo utilizzeremo i giochi di EF \mytodo{Sostituire il command}, introdotti nella Sezione \mytodo{Inserire la sezione}.

% \begin{definizione}
%   Sia $A$ una $L$-struttura.
%   Il grafo di Gaifman di $A$ è un grafo non orientato $G_A = (|A|, E^A)$, dove $G_A \models E(a,b)$ se e solo se $a$ e $b$ sono distinti e sono parte di una tupla $(c_1,\dots,c_r)$ di elementi di $A$ tali che $A\models R(c_1,\dots,c_r)$ per una qualche relazione $r$-aria $R\in L$.
% \end{definizione}
% 
% \begin{osservazione}
%   Se $L=\{E\}$ è il linguaggio dei grafi (con un'unica relazione $E$ binaria) e $A$ è un grafo non orientato, allora $G_A=A$.
% \end{osservazione}

In questa sezione, utilizzeremo il linguaggio dei grafi $L=\{E\}$, con $E$ relazione binaria.
Ci restringiamo alla classe dei grafi non orientati, ovvero i grafi $A=(|A|,E^A)$ per i quali la relazione $E^A$ è simmetrica.
Introduciamo il problema decisionale ``$\Connected$'', al quale appartengono tutti e soli i grafi connessi.
Dimostreremo che tale problema appartiene a $\coESOmon$ ma non ad $\ESOmon$.

\begin{lemma}
  $\Connected \in \coESOmon$.
\end{lemma}

\begin{proof}
  Dobbiamo esprimere la proprietà di essere sconnesso con una formula in $\ESOmon$.
  Una condizione equivalente ad essere sconnesso è che l'insieme dei vertici possa essere partizionato in due sottoinsiemi tra i quali non vi è alcun arco. Questa condizione può essere espressa mediante la seguente formula $\ESOmon$:
  \[ \exists \, U^{(1)}, W^{(1)} \Big( \forall x \big( U(x) \dot\lor W(x) \big) \,\wedge\, \forall x,y \big( E(x,y) \rightarrow (U(x) \leftrightarrow \lnot W(y))\big) \Big). \qedhere \]
\end{proof}


\begin{lemma}
  $\Connected \not\in \ESOmon$.
\end{lemma}

\begin{proof}
  Supponiamo per assurdo che esista una $L$-formula $\exists P_1^{(1)},\ldots,P_r^{(1)} \varphi$, con $\varphi\in \FO$, tale che per ogni grafo non orientato $G=(V^G, E^G)$ si abbia
  \[ G \models (\exists \vec{P}) \varphi \, \leftrightarrow G \text{ connesso}. \]
  Sia $m=\RQ(\varphi)$, e sia $\tau = \{E,P_1,\ldots,P_r\}$ un nuovo linguaggio che estende $L$ in cui le $P_i$ sono relazioni unarie.
  Osserviamo che una $\tau$-struttura può essere considerata come un grafo colorato con $2^r$ colori, dove il colore del vertice $x$ è codificato dai bit $P_1(x),\ldots,P_r(x)$.
  
  TODO: continuare
\end{proof}



\begin{teorema}
  $\ESOmon \neq \coESOmon$.
\end{teorema}




\section{$\NP$-completezza di $\SAT$}

Il Teorema~\ref{thm:np-eso} consente di dimostrare in modo particolarmente semplice l'$\NP$-com\-ple\-tez\-za del problema di soddisfacibilità booleana di formule proposizionali, denominato comunemente ``$\SAT$''.

\begin{teorema}
  $\SAT$ è $\NP$-completo.
\end{teorema}

\begin{proof}
  $\SAT$ appartiene a $\NP$ perché un certificato di soddisfacibilità di una formula è dato dal valore da assegnare alle variabili per fare in modo che la formula risulti vera.
  
  Dobbiamo dimostrare che un qualsiasi problema $\mathcal{B}\in\NP$ si riduce polinomialmente a $\SAT$.
  Supponiamo che $\mathcal{B}$ sia un insieme di strutture nel linguaggio $L=\{R_1,\dots,R_t\}$, dove $R_1,\dots,R_t$ sono relazioni di arietà $b_1,\dots,b_t$, rispettivamente.
  Per il Teorema~\ref{thm:np-eso}, esiste una formula $\gamma$ della forma
  \[ \gamma :\equiv \exists\, S_1,\dots,S_r \, \forall \underbrace{x_1,\dots,x_n}_{\text{prim'ordine}} \, \underbrace{\varphi(x_1,\dots,x_k)}_{\text{senza quantificatori}} \]
  tale che $\mathcal{B}$ sia dato dai modelli di $\gamma$.
  
  Sia $A$ una $L$-struttura, e sia $n=|A|$. Abbiamo che $A$ appartiene a $\mathcal{B}$ se e solo se soddisfa la formula $\gamma$, ovvero se e solo se esistono delle relazioni $S_1^A\subseteq n^{a_1}, \dots, S_r^A\subseteq n^{a_r}$ tali che sia soddisfatta $\forall x_1,\dots,x_n \,\varphi(x_1,\dots,x_k)$.
  Questo è equivalente a trovare un'assegnazione che renda vera la formula proposizionale
  \[ \psi :\equiv \bigwedge_{0\leq c_1,\dots,c_k < n} \varphi(c_1,\dots,c_k) \]
  nelle varabili booleane $S_i(\vec{y})$ ed $R_j(\vec{z})$, al variare di $i\in \{1,\ldots,r\}$, $j\in\{1,\dots,t\}$, $\vec{y}\in n^{a_i}$, $\vec{z}\in n^{b_j}$.
  Sostituiamo a ciascuna variabile del tipo $R_j(\vec{z})$ il valore vero o falso, a seconda che valga o meno $A \models R_i(\vec{z})$; chiamiamo $\tilde\psi$ la formula così ottenuta, che dipende ora solamente dalle variabili $S_i(\vec{y})$ al variare di $i$ e di $\vec{y}\in n^{a_i}$.
  In conclusione, $A$ appartiene a $\mathcal{B}$ se e solo se esiste un'assegnazione delle variabili $S_i(\vec{y})$ che rende vera la $\psi$. Il numero di tali variabili è $n^{a_1}+\ldots+n^{a_r}$, che è un polinomio in $n$.
  Abbiamo quindi mostrato una riduzione polinomiale di $\mathcal{B}$ a $\SAT$.
\end{proof}




