\chapter{Caratterizzazione di $\P$}

\section{Sintassi e semantica di $\FO(\LFP)$}

\begin{definizione}
 Sia $\phi(S^2,x,y)$ una formula.
 Diremo che $S$ compare in \emph{posizione positiva} in $\phi$ se tutte le volte
 in cui compare è preceduta da un
 numero pari di negazioni. Più precisamente, per induzione sulla complessità
 della formula:
 \begin{itemize}
  \item $S$ è in posizione positiva in tutte le formule atomiche (che la contengano o no)
  \item Se $S$ compare in posizione positiva in $\phi$ e $\theta$, allora compare
  in posizione positiva in $\phi \land \theta$, $\phi \lor \theta$, $\exists x \phi(x)$,
  $\forall x \phi(x)$.
  \item Se $S$ compare in posizione positiva in $\phi$ allora compare in posizione
  negativa in $\lnot \psi$, $\psi \to \phi$ e in $\lnot \psi \land \phi$.
  \item Se $S$ compare in posizione positiva in $\phi$ allora compare in posizione
  negativa in $\lnot \phi$ e viceversa se $S$ compare in posizione negativa in $\phi$
  allora compare in posizione
  positiva in $\lnot \phi$.
 \end{itemize}
\end{definizione}

\mytodo{Controllare che la definizione sia esaustiva}

\begin{definizione}
 Estendiamo la sintassi di $\FO$ introducendo un nuovo quantificatore, che
 indichiamo con $\LFP$. Data una
 formula del primo ordine $\phi(S^2,x,y)$ in cui $S$ compare in posizione positiva,
 l'espressione $\LFP_{S,x,y} \phi(S^2,x,y)$ avrà il tipo di una relazione binaria
 con variabili
 libere $\FV(\phi) \setminus \{S,x,y\}$.
 Chiameremo la sintassi così ottenuta $\FO(\LFP)$.
\end{definizione}

Semanticamente vogliamo interpretare $\LFP_{S,x,y} \phi(S,x,y)$ come la più
piccola (rispetto all'inclusione) relazione binaria $R$ tale che $R(x,y) \leftrightarrow \phi(R,x,y)$, ma dobbiamo
prima verificare che esiste una tale relazione.

\begin{teorema}[Tarski-Knaster]
\label{teo-tarski-knaster}
 Sia $A$ un insieme finito e sia $f: \powerset(A^k) \to \powerset(A^k)$ una
 funzione monotona rispetto all'inclusione
 (i.e. se $S \subseteq S'$ allora $f S \subseteq f S'$). Allora esiste un minimo
 punto fisso di $f$, che indicheremo con $\LFP f$ e per la precisione
 questo è dato da $f^{|A|^k}(\varnothing)$.
\end{teorema}

\begin{proof}
 Dato che $f$ è monotona e $A$ è finito, deve esistere $n$ in $\N$ tale che
 $f^n(\varnothing) = f^{n+1}(\varnothing)$, e inoltre tale $n$ può essere preso
 minore di $|A|^k$. Indichiamo $f^n(\varnothing)$ con $\bar{S}$.
 Per definizione $\bar{S}$ è un punto fisso di $f$.
 Se $S'$ è un altro punto fisso di $f$, dato che $\varnothing \subseteq S'$ e
 che $f$ è monotona, avremo $\bar{S} = f^k(\varnothing) \subseteq f^k(S') = S'$.
 Quindi $\bar{S}$ è il più piccolo punto fisso.
\end{proof}

\mytodo{Accennare al caso infinito}


\begin{corollario}
\label{cor-f-phi}
 Sia $\phi(S,x,y)$ una $L$-formula del primo ordine in cui $S$ compare in posizione
 positiva, e sia $A$ una $L$-struttura.
 Indichiamo con $f_\phi: \powerset (A \times A) \to \powerset (A\times A)$ la funzione
 \[ f_{\phi,A}(R) := \{ (a,b) \in A \times A \mid \semantic{ \phi (S,x,y)}^{A,R,a,b} \}. \]
 Questa funzione è monotona sulle relazione ordinate per inclusione.
 Il suo più piccolo punto fisso $\LFP f_{\phi,A}$ è la più piccola relazione
 $R \subseteq A \times A$ tale che $R(a,b) \leftrightarrow \semantic{\phi(S,x,y)}^{A,R,a,b}$.
\end{corollario}
\mytodo{Dimostrazione}

\begin{definizione}
 Usando le notazioni del Corollario \ref{cor-f-phi}, estendiamo la funzione
 semantica di $\FO$ ad $\FO(\LFP)$ ponendo
 \[\semantic{\LFP_{S,x,y} \phi(S,x,y)}^M = \LFP f_{\phi,M}.\]
\end{definizione}

\section{$\FO(\LFP)$ equivale a $\P$}

\begin{lemma}
 $\FO \subseteq \P$, ovvero, per ogni $L$-formula chiusa $\phi$ in $\FO$ esiste
 una macchina di Turing $T_\phi$ in $\P$ tale che, data una $L$-struttura $A$,
 si ha $A \models \phi$ se e solo se $T_\phi(\bin(A))\downarrow$. Inoltre
 le $T_\phi$ possono essere costruite in modo che siano logaritmiche nello spazio.
\end{lemma}
\begin{proof}
 \mytodo{Pseudocodice da mettere ovunque!}
 Procediamo per induzione strutturale sulla complessità della formula $\phi$.
 
 Sia $\phi$ atomica nel linguaggio $L=\{ R^{a_1},\ldots,R^{a_k},c_1,\ldots,c_l \}$,
 allora $\phi$ deve essere nella forma $c_1=c_2$ oppure nella forma $R_s(c_1,\ldots, c_k)$,
 dove i $c_i$ sono simboli di costante (ricordiamo che per convenzione
 non ammettiamo simboli di funzione nel linguaggio). Il nastro di input della
 macchina di Turing $T_\phi$ avrà la seguente struttura
 \[ \ldots \]
 La macchina di Turing $T_\phi$ come prima cosa leggerà dal nastro di
 input i valori $m_i$ corrispondenti alle costanti $c_i$.
 Se $\phi$ era nella forma $c_1 = c_2$ allora basta controllare l'uguaglianza
 di $m_1$ ed $m_2$. Se invece $\phi$ era nella forma $R_s(c_1,\ldots,c_k)$,
 leggiamo il bit in posizione $m_1 + m_2 \cdot n + \ldots + m_k \cdot n^{k-1}$
 del blocco relativo alla relazione $R_k$ e accettiamo nel caso abbia valore 1.
 
 Se $\phi$ è combinazione booleana di formule $\phi_i$,
 la macchina di Turing corrispondente si ottiene facilmente mettendo insieme le
 macchiene di Turing delle singole $\phi_i$.
 
 Sia infine $\phi$ nella forma $\exists x \theta(x)$. Detto $L'$ il linguaggio
 ottenuto aggiungendo a $L$ un simbolo di costante $c$,
 la $L'$-formula $\theta(c)$ ha complessità minore di quella di $\phi$
 e dunque, per ipotesi induttiva, esiste una
 macchina di Turing $T_{\theta(c)}$ che gli corrisponde.
 Notiamo che, dato $x$ in $\{0,\ldots, n-1\}$, possiamo trasformare $A$ in una $L'$
 struttura $(A,x)$ interpretando la costante $c$ con $x$.
 Costruiamo allora la macchina di Turing $T_\phi$ in modo che $T_\phi(\bin(A))$
 accetti se e solo se $T_{\theta(c)}(\bin(A,x))$ accetta per qualche $x$.
 \mytodo{Le macchine che costruiamo si aspettavano di trovare l'input
 sul nastro di input, ma io non posso modificare il nastro di input per aggiungere $\bin(x)$.}
 \mytodo{Forse è meglio metterci lo pseudo-codice}.
\end{proof}

\begin{proposizione}
 $\FO(\LFP) \subseteq \P$, ovvero, per ogni $L$-formula chiusa $\phi$ in $\FO(LFP)$ esiste
 una macchina di Turing $T_\phi$ in $\P$ tale che, data una $L$-struttura $A$,
 si ha $A \models \phi$ se e solo se $T_\phi(\bin(A))\downarrow$.
\end{proposizione}
\begin{proof}
  Rispetto al lemma precedente dobbiamo solo considerare in più il caso
  $\phi \equiv \LFP_{S,x,y}\theta(S,x,y)(a,b)$, con $a, b$ costanti.
  Notiamo che possiamo vedere
  $\theta(S,x,y)$ come una formula chiusa nel linguaggio $L'=L \cup \{S^2,x,y\}$,
  dunque ha senso considerare la macchina di Turing $T_{\theta(S,x,y)}$ ad essa
  corrispondente. Inoltre, usando le notazioni del Corollario \ref{cor-f-phi},
  si ha $T_{\theta(S,x,y)}(\bin(A,R,a,b))\downarrow$
  se e solo se $(a,b) \in f_{\theta,A}(R)$. Questa osservazione, insieme alla
  dimostrazione del Teorema \ref{teo-tarski-knaster}, suggeriscono il seguente
  algoritmo per $T_\phi$ e ne dimostrano la correttezza.
%   Riserviamo una zona $R$ in memoria di grandezza $|A|^2$, inizializziamola a zero e
%   per $|A|^2$ passi iteriamo il seguente procedimento: per ogni coppia $(a,b)$ in $A^2$,
%   eseguiamo la macchina $T_{\theta(S,x,y)}(\bin(A,R,a,b))$ e se accetta settiamo a 1 il bit
%   in posizione $a \cdot n + b$ del blocco $R$.
  
  \begin{algorithmic}
  \Function{$T_\phi$}{$bin$}
    \State $R\gets 0$
    \For {$i < max^2$}
      \For {$h,k = 0..max$}
      \If {$T_{\theta(S,x,y)}(\bin(A)\string^\bin(h)\string^\bin(k))\downarrow$}
	\State $R[a*max + b] \gets 1$
      \EndIf
      \EndFor
    \EndFor
    \State $a \gets \decode_a (bin)$
    \State $b \gets \decode_b (bin)$
    \State \Return $R[a*max + b]$
  \EndFunction
  \end{algorithmic}
\end{proof}

Dimostriamo ora l'inclusione inversa, ovvero $\P \subseteq \FO(\LFP)$. La
dimostrazione si basa sui seguenti claim:
\begin{description}
 \item[Claim 1]\label{p_eq_a} $\P = \ASPACE(\lg(n))$;
 \item[Claim 2]\label{ralt_in_p} $\Ralt \in \FO(\LFP)$;
 \item[Claim 3]\label{reduct_fo} Se $S$ è un problema in $\ASPACE(\lg(n))$, allora $S \leq_{\FO} \Ralt$.
 Equivalentemente (assumendo di aver dimostrato il Claim 1), $\Ralt$ è un problema $P$-completo.
\end{description}
Dimostrati questi tre claim, per concludere, basta notare che un problema che si
riduce $\FO$ a un problema in $\FO(\LFP)$ è a sua volta in $\FO(\LFP)$.
Il Claim 1 segue dal Teorema \ref{thm:aspace-dtime} prendendo $t(n) = \ln(n)$.
Dimostriamo dunque gli altri due.
\begin{proposizione}
  $\Ralt \in \FO(\LFP)$;
\end{proposizione}
\begin{proof}
 \mytodo{Da scrivere, ma prima è meglio scrivere la definizione di macchina di
 Turing alternante}
\end{proof}

\begin{proposizione}
 Se $S$ è un problema in $\ASPACE(\lg(n))$, allora $S \leq_{\FO} \Ralt$.
\end{proposizione}
\begin{proof}
 Sia $\sigma = \{R^{a_1}_1,\ldots,R^{a_r}_r,c_1, \ldots, c_s\}$ un linguaggio
 e $M$ una macchina di Turing in $\ASPACE(c \lg(n))$ che accetta come input
 la codifica binaria di una $\sigma$-struttura. Dobbiamo mostrare che esiste un
 modo per associare ad ogni $\sigma$-struttura $A$ di cardinalità $n$ un grafo alternante
 $I(A) = (V^A, G_\exists^A, G_\forall^a, E^a, s^A,t^A)$,
 definibile al primo ordine in $A$, tale che:
 \[ M(\bin(A)) \iff I(A) \in \Ralt. \]
 L'idea è di associare alla struttura $A$ il grafo della computazione
 di $M$ su input $\bin(A)$, in questo modo la nostra richiesta sarà
 automaticamente soddisfatta.
 A tal fine identifichiamo il dominio di $A$ con l'insieme $\{1,2,\ldots, n\}$ e
 scegliamo come insieme di vertici $V^A$ l'insieme $A^{4+a+c}$,
 con $a=\max(a_1, \ldots, a_r)$. In questo modo
 un vertice $\ID$ del grafo può essere identificato con una tupla di numeri naturali
 minori di $n$:
 \[\ID=\langle p_1, \ldots, p_4, r_1, \ldots, r_a, w_1, \ldots, w_c \rangle\]
 Ciascuna di queste tuple rappresenterà una possibile configurazione della
 macchina $M$, per la precesione:
 \begin{itemize}
  \item $p_1 \in \{1,\ldots, r\}$ rappresenta la relazione che la
  macchina sta leggendo sul nastro di input, come verrà spiegato più tardi;
  \item $p_2 \in Q_M$ rappresenta lo stato in cui si trova la macchina di Turing;
  \item $p_3 \in \{0, \ldots, c \lg(n)\}$ rappresenta la posizione della testina
  sul nastro di lavoro;
  \item $p_4 \in \{\forall, \exists\}$ indica se lo stato attuale è esistenziale
  o universale;
  \item $\langle w_1, \ldots, w_c \rangle \in n^c = 2^{c \lg (n)}$ rappresenta
  una possibile configurazione del nastro di lavoro (stiamo usando
  il fatto che $M$ è in $\ASPACE(c \ln(n))$;
  \item $\langle r_1, \ldots, r_c \rangle \in n^c = r_1 + r_2 n + \ldots + r_a n^{a-1} < n^a$
  rappresenta la posizione della testina sul nastro di input.
 \end{itemize}
  
  Ora che abbiamo definito il dominio $V^A$ del grafo dobbiamo definire le
  relazioni $E^A$, $G_\exists^A$, $G_\forall^A$, $s$ e $t$. Di queste, $G_\exists$
  e $G_\forall$ sono immediate
  (basta guardare il valore di $p_4$), per quanto riguarda $S$, può essere
  definita come
  \[s(\ID) \equiv (p_1=1 \land p_2 = q_{init} \land p_3=0 \land \vec{r}=0 \land \vec{w}=0),\]
  e anche $t$ può essere definita in maniera simile.
  Concentriamoci dunque su $E^A$: dobbiamo
  trovare una relazione $E(\vec{x},\vec{y})$ tale che
  valga $E(\ID, \ID')$ se e solo se nel grafo delle
  computazioni di $M$ con input $\bin(A)$ c'è un arco tra la configurazione
  corrisponente a $\ID$ quella corrispondente a $\ID'$.
  Come prima cosa ad ogni regola $R$ della macchina di Turing $M$ nella forma
  \[\langle q_R, T_R, wk_R, in_R \rangle \vdash
   \langle q'_R, T'_R, out_R, wksh_R, insh_R \rangle
  \]
  associamo la relazione
  \[C_R(\ID, \ID') \equiv \ldots \]
  Notiamo che vale $C_R(\ID, \ID')$ se e solo se la configurazione corrispondente
  a $\ID$ può passare alla configurazione corrispondente a $\ID'$ mediante la regola
  $R$. Per concludere definiamo allora
  \[ E(\ID, \ID') \equiv \bigvee_R C_R(\ID, \ID'). \]
\end{proof}

\begin{osservazione}
  Nella dimostrazione precedente potevamo
  memorizzare tutte le possibili configurazione del nastro di lavoro all'interno
  del grafo grazie all'ipotesi $M \in \ASPACE(c \lg(n))$, tuttavia non possiamo
  fare lo stesso per il nastro di input che ha lunghezza
  polinomiale in $n$ (e quindi il numero di possibili configurazioniè esponenziale in $n$).
  Tuttavia, come visto, possiamo limitarci a memorizzare soltanto la posizione
  della testina sul nastro di input.
\end{osservazione}

\begin{corollario}
 $\P \subseteq \FO(\LFP)$, ovvero, per ogni macchina di Turing $M$ che accetta o
 rifiuta una $\sigma$-struttura $A$ in tempo polinomiale, esiste una
 $\sigma$-formula $\phi$ in $\FO(\LFP)$ tale che:
 \[ A \models \phi \iff M(\bin(A))\downarrow \]
\end{corollario}



