\chapter{Classi di complessità}

\section{ATIME}

\begin{teorema}
 $\ATIME(t(n)) \subseteq \DSPACE(t(n))$
\end{teorema}
\begin{proof}
 Sia $A$ una macchina di Turing alternante in $\ATIME(t(n))$, vogliamo trovare
 una macchina di Turing deterministica $B$ in $\DSPACE(t(n))$ che accetti gli
 stessi input $w$ di $A$. Supponiamo per semplicità di notazione che ogni stato
 $A$ non deterministico di $A$ possa fare una transizione in al più due stati.
 Sia $G_w$ il grafo delle computazioni della macchina $A$ quando riceve input $w$.
 Per definizione un nodo del grafo è accettante se:
 \begin{itemize}
  \item è esistenziale e almeno uno dei suoi figli è accettante, oppure
  \item è universale e entrambi i suoi figli sono accettanti.
 \end{itemize}
 Esibiamo una macchina di Turing in $\DSPACE(t(n))$ che, preso in input un nodo
 del grafo, decide se quel nodo è accettante o meno. Questo concluderebbe, infatti
 la macchina $A$ accetta l'input $w$ se e solo se il nodo iniziale
 di $G_w$ è accettante.
 
 La macchina $B$ deve verificare se il nodo iniziale di $G_w$ sia accettante o
 meno. Per far questo deve visitare ricorsivamente il grafo, tuttavia questo
 è troppo grande per essere tenuto tutto in memoria. La soluzione che adottiamo
 è di memorizzare solo il nodo iniziale $c_0$, il nodo $c$ che stiamo attualmente
 visitando e il percorso che abbiamo fatto per raggiungerlo a partire dal nodo
 iniziale, ovvero una stringa binaria $\langle b_1 b_2 \ldots b_n \rangle$
 che ci dica per ogni nodo alternante
 che abbiamo incontrato se dobbiamo spostarci sul primo o sul secondo figlio.
 Teniamo inoltre in memoria un simbolo $D \in \{S, N, ?\}$ che ci dice se
 abbiamo già verificato che quel nodo è accettante, non accettante oppure
 se ancora non lo sappiamo.
 
 
 
 
\end{proof}

